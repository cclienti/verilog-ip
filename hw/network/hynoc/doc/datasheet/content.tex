\section{Description}
\subsection{Architecture}

A router is made of full-duplex ports which rely on ingress and egress interfaces and that are respectively plugged to
egress and ingress of another router ports. Each router has its own clock domains and the crossing rely on dual-clocked
FIFO at the input of each ingress interface. A node is attached to a router using a local interface, this interface
instantiates an extra FIFO to the egress output to support the node's clock domain.

Figure \ref{hynoc_noc_overview} shows both internal organization of 5-port routers and how they are connected to their
neighbors.

\begin{figure}[H]
  \centering
  \includegraphics[width=0.8\linewidth]{figures/hynoc_noc_overview.pdf}
  \caption{Router interconnections overview}
  \label{hynoc_noc_overview}
\end{figure}

The router is a full crossbar, meaning that each ports can establish a communication to all ports except with
itself. Multiple paths inside the router can be opened at the same time thanks to the distributed arbitration
scheme. Thus, each egress port embeds an output multiplexer and an arbiter to route, without starvation, data from the
ingress ports that has requested a transfer.

Figure \ref{hynoc_router_data_arch} presents the data paths between ingress and egress. Each ingress broadcasts its data
to all egress ports except to the one which is grouped in the same router port. Then, the egress will forward data to
the next router depending on the request asserted by the ingress ports and also depending on the state of the arbiter.

\begin{figure}[H]
  \centering
  \includegraphics[width=0.8\linewidth]{figures/hynoc_router_data_arch.pdf}
  \caption{Internal data path of a 3-port router}
  \label{hynoc_router_data_arch}
\end{figure}

The control path is shown in the figure \ref{hynoc_router_control_arch}. Contrary to the data path structure, there is
no broadcast of control signals. Each ingress has dedicated links with each egress to assert transmit requests and to
receive the grant from an egress. Once the grand is received, the ingress can push data.

\begin{figure}[H]
  \centering
  \includegraphics[width=0.8\linewidth]{figures/hynoc_router_control_arch.pdf}
  \caption{Internal control path of a 3-port router}
  \label{hynoc_router_control_arch}
\end{figure}

The egress arbiter uses a parallel round-robin arbiter which allows to schedule all ingress requests without starvation
in a fix latency.

\section{First Layer Protocol}

\subsection{Packet routing}

HyNoC uses source routing techniques to send data through routers. Instead of addressing node with coordinates, for
instance (x,y) for a network with a mesh topology, we define in the packet's header the route to take through the
network. This means that the header contains a list of output ports, called hops, of routers to cross. This technique is
used by \cite{LIW2007} and \cite{MUB2010} to avoid congestion with a low packet header overhead using a simple encoding
hop scheme.

A routing algorithm can be defined upon Source Routing network depending on topology. \cite{MUB2010} survey presents
such a technique. The routing algorithm generates for each communication the list of hops either dynamically (in
hardware in the local interface or in software using the node's processor) or statically at compilation time.

Routes are established in a distributed way in each router, this technique is presented by \cite{PON2011}. Each egress
interface manage itself and it is in charge of selecting the right ingress interface using a simple request/acknowledge
protocol. Moreover, the egress interface embeds a LUT-based parallel round robin arbiter to respond in fix amount of
time without creating any starvation.


\subsection{Packet structure}

A packet is composed of multiple flits, a flit is the smallest unit transmitted over the network and it is composed of
$K+1$ bits. The most significant bit is the last bit and it indicates the last flit of a packet.

The table \ref{packet_definition} shows the packet structure. A packet is built with at least one Routing flit followed
by at least one Payload flit. If multiple header flits are used to encode the path through the network, the router will
consider only the first flit as a header and the remaining flits as payload. When all hops of the header are marked
used, that header flit will not be forwarded to the next router. Thus, the next router will use the next flit as a
header flit and so on.

\begin{table}[h]
  \centering
  \begin{tabular}{c | c | c}
    \toprule\hline
    \textbf{Last bit} & \multicolumn{2}{c}{\textbf{payload (K bits)}} \\
    \hline\hline
    0 & 4-bit Proto & Routing flit \\
    \hline
    0 & ... & ... \\
    \hline
    0 & \multicolumn{2}{c}{Payload flit} \\
    \hline
    0 & \multicolumn{2}{c}{...} \\
    \hline
    1 & \multicolumn{2}{c}{last Payload flit} \\
    \hline\bottomrule
  \end{tabular}
  \caption{Packet definition}
  \label{packet_definition}
\end{table}


The table \ref{smallest_packet} presents the smallest packet that can be sent over the network. The number of router
that it can pass through depends on the protocol, the payload width and the number of ports within a router.

\begin{table}[h]
  \centering
  \begin{tabular}{c | c | c}
    \toprule\hline
    \textbf{Last flit bit} & \multicolumn{2}{c}{\textbf{payload (K bits)}} \\
    \hline\hline
    0 & 4-bit Proto & Routing flit \\
    \hline
    1 & \multicolumn{2}{c}{last Payload flit} \\
    \hline\bottomrule
  \end{tabular}
  \caption{Smallest packet}
  \label{smallest_packet}
\end{table}


\subsection{Routing protocols}

Multiple protocols can be supported in a routing flit depending on the 4-bit Proto value. The table
\ref{supported_protocol} shows supported protocols and the following sub-sections describe how they work.


\begin{table}[h]
  \centering
  \begin{tabular}{l | c}
    \toprule\hline
    \textbf{Proto value} & \textbf{Routing method} \\
    \hline\hline
    4'b0000 & Unicast Circuit Switch \\
    \hline
    4'b0001 & Multicast Circuit Switch \\
    % 4'b1000 & XY routing\\
    \hline\bottomrule
  \end{tabular}
  \caption{Supported routing protocols}
  \label{supported_protocol}
\end{table}


\subsubsection{Unicast Circuit Switch routing}

In this routing policy, the routing flit is a list of router egress ID to cross. The table
\ref{unicast_circuit_switch_field} describes the unicast circuit switch protocol field. For a router with P ports, the
Hop field is encoded using $W=\lceil \log_2(P-1)\rceil$ bits. The index field points the correct hop to be used by the
router ingress port. The Gap width can be in $[0, W-1]$.

\begin{table}[h]
  \centering
  \begin{tabular}{c | c | c | c | c | c}
    \toprule
    \hline
    \textbf{Proto} & \multicolumn{5}{c}{\textbf{Hops list (K-4 bits)}} \\
    \hline\hline
    4'b0000 & Gap & Hop H-1 & ... & Hop 0 & Index \\
    \hline
    \bottomrule
  \end{tabular}
  \caption{Unicast Circuit Switch Field Description}
  \label{unicast_circuit_switch_field}
\end{table}

The index is numbered between $[0, H-1]$ for a flit with $H$ hops and is initialized to $H-1$. The index is decremented
once the pointed hop is used to open the path inside the router. If the index is null before path opening, the related
Network Hops flit will not be transmitted to the router's egress port, else the index is decremented and the updated
Routing flit is transmitted. A unicast flit can embed less than the maximum allowed number of hop fields by just
initializing the index accordingly.

The hop encoding is based upon the fact that a same packet can not use the same port for incoming and outgoing. The
number of accessible egress port are $P-1$ for a router with $P$ ports. This technique reduces the internal router's
crossbar size.

The figure \ref{hynoc_hops_encoding} shows the hops list $0 \rightarrow 2 \rightarrow 0 \rightarrow 1 \rightarrow 2$ to
establish a communication channel from node $(0, 0)$ to node $(2,2)$ in a $3 \times 3$ mesh network. This NoC is built
upon 5-port routers and only two bits are needed to encode each hop.

\begin{figure}[H]
  \centering
  \includegraphics[width=0.6\linewidth]{figures/hynoc_hops_encoding.pdf}
  \caption{Unicast Circuit Switch hops encoding}
  \label{hynoc_hops_encoding}
\end{figure}

Each ports encodes the next counterclockwise egress with the ID zero. In other words, the selected egress ID must be
calculated by taking into account the ingress id, this is a relative egress addressing.

An example of an unicast packet is given in figure \ref{hynoc_unicast_packet_example_64b}. It corresponds to the path
described in the figure \ref{hynoc_hops_encoding}, the NoC payload width is 64-bit.

\begin{figure}[H]
  \centering
  \includegraphics[width=0.7\linewidth]{figures/hynoc_unicast_packet_example_64b.pdf}
  \caption{Example of a unicast circuit switch packet of a 64-bit NoC}
  \label{hynoc_unicast_packet_example_64b}
\end{figure}

\begin{figure}[H]
  \centering
  \includegraphics[width=0.5\linewidth]{figures/hynoc_unicast_packet_example_16b.pdf}
  \caption{Example of a unicast circuit switch packet of a 16-bit NoC}
  \label{hynoc_unicast_packet_example_16b}
\end{figure}

\subsubsection{Multicast Circuit Switch routing}

The multicast routing policy permits to target multiple egress ports at a time from a unique ingress port. The table
\ref{multicast_circuit_switch_field} describes the multicast circuit switch protcol field. For a router with P ports,
the Hop field width is $W=P-1$ bits. The index field points the correct hop to be used by the router ingress port. The
Gap width can be in $[0, W-1]$.

As a reminder, the index must be initialized to $H-1$, because the ingress port forwards the flit by decrementing the
index. A multicast flit can embed less than the maximum allowed number of hop fields by just initializing the index
accordingly.

\begin{table}[h]
  \centering
  \begin{tabular}{c | c | c | c | c | c}
    \toprule
    \hline
    \textbf{Proto} & \multicolumn{5}{c}{\textbf{Hops list (K-4 bits)}} \\
    \hline\hline
    4'b0001 & Gap & Hop H-1 & ... & Hop 0 & Index \\
    \hline
    \bottomrule
  \end{tabular}
  \caption{Multicast Circuit Switch Field Description}
  \label{multicast_circuit_switch_field}
\end{table}

\subsubsection{Combining multiple routing policies}

Multiple types of routing flits can be mixed at the beginning of a packet.

\section{Router example: 32-bit 3-port}

\subsection{introduction}

The 32-bit 3-port router is delivered with a testbench that demonstrates the functionality and the behavior of the
router. The following subsections describe the packet structure used, the module parameters and ports description and an
overview of the testing environment proposed.

\subsection{Packet structure}

The 32-bit 3-port NoC can hold up to 23 hops for unicast packets and up to 11 hops for multicast packets. The figures
\ref{hynoc_unicast_packet_3port_32b} and \ref{hynoc_mcast_packet_3port_32b} describe the packets structures for both
unicast and multicast protocol.

\begin{figure}[H]
  \centering
  \includegraphics[width=0.7\linewidth]{figures/hynoc_unicast_packet_3port_32b.pdf}
  \caption{Unicast packet structure for 3-port 32-bit NoC}
  \label{hynoc_unicast_packet_3port_32b}
\end{figure}

\begin{figure}[H]
  \centering
  \includegraphics[width=0.7\linewidth]{figures/hynoc_mcast_packet_3port_32b.pdf}
  \caption{Multicast packet structure for 3-port 32-bit NoC}
  \label{hynoc_mcast_packet_3port_32b}
\end{figure}


\subsection{Parameters}

The following module parameters declared in the 3-port router module are presented in the following table
\ref{parameters_3port_router_32bit}. Please note that some can be directly inlined, depending on the delivery content.

\begin{table}[h]
  \centering
  \begin{tabular}{l|l|p{7cm}}
    \toprule
    \hline
    \textbf{Name} & \textbf{Default Value} & \textbf{Description}\\
    \hline\hline
    INDEX\_WIDTH & 5 & Bit width of the index in an address flit. \\
    \hline
    LOG2\_FIFO\_DEPTH & 5 & Size of the FIFO inserted in each port's ingress expressed in $log_2$ basis. \\
    \hline
    PAYLOAD\_WIDTH & 32 & Bit width of the payload. \\
    \hline
    FLIT\_WIDTH & $\text{PAYLOAD\_WIDTH}+1$ & Bit width of the flit. \\
    \hline
    PRRA\_PIPELINE & 0 & 2-cycle parallel round-robin arbiter response when set to 0 else 3-cycle. \\
    \hline
    SINGLE\_CLOCK\_ROUTER & 0 & When set to 1, each port uses the router clock instead of its own clock to reduce
                                the traversal latency. \\
    \hline
    ENABLE\_MCAST\_ROUTING & 1 & When set to 1, enable the multicast routing protocol. \\
    \hline
    \bottomrule
  \end{tabular}
  \caption{Parameters of the 3-port 32-bit router}
  \label{parameters_3port_router_32bit}
\end{table}


\subsection{Ports}

The router ports description is presented in \ref{ports_3port_router_32bit}. The port is composed of two parts, an
ingress side and an egress side. The ingress side receives data and the egress side sends data. When connecting two
ports, the egress port X must be connected to ingress port Y and vice-versa.

\begin{table}[h]
  \centering
  \begin{tabular}{l|l|l|p{7cm}}
    \toprule
    \hline
    \textbf{Name} & \textbf{Dir} & \textbf{Width} & \textbf{Description}\\
    \hline\hline
    router\_clk & In & 1 & Internal router clock. \\
    \hline
    router\_srst & In & 1 & Internal router synchronous reset. \\
    \hline
    portX\_ingress\_clk & In & 1 & Clock of ingress port X,
                                   not used if SINGLE\_CLOCK\_ROUTER is set to 1. \\
    \hline
    portX\_ingress\_srst & In & 1 & Synchronous reset of the ingress port X,
                                    not used if SINGLE\_CLOCK\_ROUTER is set to 1. \\
    \hline
    portX\_ingress\_write & In & 1 & Send the word in the ingress port X.
                                     Must be set only one cycle. \\
    \hline
    portX\_ingress\_data & In & FLIT\_WIDTH & Data to send to the ingress port X. \\
    \hline
    portX\_ingress\_full & Out & 1 & Full signal of the ingress port X.
                                     If full, the write request is ignored. \\
    \hline
    portX\_ingress\_fifo\_level & Out & LOG2\_FIFO\_DEPTH+1 & FIFO level of the ingress port X.\\
    \hline
    portX\_egress\_clk & In & 1 & Clock of egress port X,
                                  not used if SINGLE\_CLOCK\_ROUTER is set to 1. \\
    \hline
    portX\_egress\_srst & In & 1 & Synchronous reset of the egress port X,
                                   not used if SINGLE\_CLOCK\_ROUTER is set to 1. \\
    \hline
    portX\_egress\_write & Out & 1 & A data is ready on egress port X,
                                     set one cycle per data to read. \\
    \hline
    portX\_egress\_data & Out & FLIT\_WIDTH & Data to read on egress port X.\\
    \hline
    portX\_egress\_fifo\_level & In & LOG2\_FIFO\_DEPTH+1 & Input FIFO level on egress port X, it must be connected
                                                            to another router. \\
    \hline
    \bottomrule
  \end{tabular}
  \caption{Ports of the 3-port 32-bit router}
  \label{ports_3port_router_32bit}
\end{table}

\subsection{Local Interface}

A local interface can be used in order to connect a client to a router port. The local interface instantiates an
extra FIFO to the egress router port in order to let the client sends an receives data at its own pace.

The local interface exposes a router side that connects to the egress/ingress router port and a local side that exposes
FIFO interfaces to the client in order to send and receive data. The tables
\ref{router_local_interface_port_description} and \ref{client_local_interface_port_description} presents the local
interface ports.

\begin{table}[h]
  \centering
  \begin{tabular}{l|l|l|p{7cm}}
    \toprule
    \hline
    \multicolumn{4}{l}{\textbf{Router side}} \\
    \hline
    \textbf{Name} & \textbf{Dir} & \textbf{Width} & \textbf{Description}\\
    \hline\hline
    port\_ingress\_srst & Out & 1 & \\
    \hline
    port\_ingress\_clk  & Out & 1 & \\
    \hline
    port\_ingress\_write & Out & 1 & \\
    \hline
    port\_ingress\_data & Out & FLIT\_WIDTH & \\
    \hline
    port\_ingress\_full & In & 1 & \\
    \hline
    port\_ingress\_fifo\_level & In & LOG2\_FIFO\_DEPTH + 1 & \\
    \hline
    port\_egress\_srst & In & 1 & \\
    \hline
    port\_egress\_clk & In & 1 & \\
    \hline
    port\_egress\_write & In & 1 & \\
    \hline
    port\_egress\_data & In & FLIT\_WIDTH & \\
    \hline
    port\_egress\_fifo\_level & Out & LOG2\_FIFO\_DEPTH+1 & \\
    \hline
  \end{tabular}
  \caption{Local interface port description - Router side}
  \label{router_local_interface_port_description}
\end{table}

\begin{table}[h]
  \centering
  \begin{tabular}{l|l|l|p{7cm}}
    \toprule
    \hline
    \multicolumn{4}{l}{\textbf{Client side}} \\
    \hline
    \textbf{Name} & \textbf{Dir} & \textbf{Width} & \textbf{Description}\\
    \hline\hline
    local\_clk & In & 1 & \\
    \hline
    local\_srst & In & 1 & \\
    \hline
    local\_ingress\_write & In & 1 & \\
    \hline
    local\_ingress\_data & In & FLIT\_WIDTH & \\
    \hline
    local\_ingress\_full & Out & 1 & \\
    \hline
    local\_ingress\_fifo\_level & Out & LOG2\_FIFO\_DEPTH + 1 & \\
    \hline
    local\_egress\_read & In & 1 & \\
    \hline
    local\_egress\_data & Out & FLIT\_WIDTH  & \\
    \hline
    local\_egress\_empty & Out  & 1 & \\
    \hline
    local\_egress\_fifo\_level & Out & LOG2\_FIFO\_DEPTH + 1 & \\
    \hline
    \bottomrule
  \end{tabular}
  \caption{Local interface port description - Client side}
  \label{client_local_interface_port_description}
\end{table}

\subsection{Test environment}

Two tests are delivered with the 3-port 32-bit NoC to illustrate the two possible communication modes: unicast and
multicast. Both tests use the same network topology as depicted in the figure \ref{hynoc_test_topology_3port}. The
design proposed here consist in 2 routers connected with 4 nodes and each node is connected to the router using a local
interface (LI).

\begin{figure}[H]
  \centering
  \includegraphics[width=0.8\linewidth]{figures/hynoc_test_topology_3port.pdf}
  \caption{Unicast test for 3-port 32-bit routers}
  \label{hynoc_test_topology_3port}
\end{figure}

\paragraph{Unicast}

The first test is related to unicast transfers, where each node send packets to all other nodes. The hops encoded in the
routing flits can be established using the table \ref{unicast_routing_table_3port_32bit_router} and all node dialog
combinations are listed hereafter:

\begin{itemize}
\item Node 0 $\rightarrow$ Node 1: hops $[0]$
\item Node 0 $\rightarrow$ Node 2: hops $[1 \rightarrow 0]$
\item Node 0 $\rightarrow$ Node 3: hops $[1 \rightarrow 1]$
\end{itemize}
\begin{itemize}
\item Node 1 $\rightarrow$ Node 0: hops $[1]$
\item Node 1 $\rightarrow$ Node 2: hops $[0 \rightarrow 0]$
\item Node 1 $\rightarrow$ Node 3: hops $[0 \rightarrow 1]$
\end{itemize}
\begin{itemize}
\item Node 2 $\rightarrow$ Node 0: hops $[1 \rightarrow 0]$
\item Node 2 $\rightarrow$ Node 1: hops $[1 \rightarrow 1]$
\item Node 2 $\rightarrow$ Node 3: hops $[0]$
\end{itemize}
\begin{itemize}
\item Node 3 $\rightarrow$ Node 0: hops $[0 \rightarrow 0]$
\item Node 3 $\rightarrow$ Node 1: hops $[0 \rightarrow 1]$
\item Node 3 $\rightarrow$ Node 2: hops $[1]$
\end{itemize}

\begin{table}[h]
  \centering
  \begin{tabular}{c | c }
    \toprule
    \hline
    \textbf{Direction} & \textbf{Unicast} \\
    \hline\hline
    P0 $\rightarrow$ P1 & 1'b0 \\
    \hline
    P0 $\rightarrow$ P2 & 1'b1 \\
    \hline
    P1 $\rightarrow$ P2 & 1'b0 \\
    \hline
    P1 $\rightarrow$ P0 & 1'b1 \\
    \hline
    P2 $\rightarrow$ P0 & 1'b0 \\
    \hline
    P2 $\rightarrow$ P1 & 1'b1 \\
    \hline
    \bottomrule
  \end{tabular}
  \caption{Unicast routing table of 3-port 32-bit router}
  \label{unicast_routing_table_3port_32bit_router}
\end{table}

\paragraph{Multicast}

The second test is a multicast transfer from node 0 to node 2 and node 3 and simultaneously from node 3 to node 0 and 1.
The address flits can be deduced using the routing table \ref{multicast_routing_table_3port_32bit_router} and are
presented hereafter:

\begin{itemize}
\item Node 0 $\rightarrow$ (Node 2, Node 3): hops $[b10 \rightarrow b11]$
\item Node 3 $\rightarrow$ (Node 0, Node 1): hops $[b01 \rightarrow b11]$
\end{itemize}

\begin{table}[h]
  \centering
  \begin{tabular}{c | c }
    \toprule
    \hline
    \textbf{Direction} & \textbf{Multicast} \\
    \hline\hline
    P0 $\rightarrow$ P1 & 2'bx1 \\
    \hline
    P0 $\rightarrow$ P2 & 2'b1x \\
    \hline
    P1 $\rightarrow$ P2 & 2'bx1 \\
    \hline
    P1 $\rightarrow$ P0 & 2'b1x \\
    \hline
    P2 $\rightarrow$ P0 & 2'bx1 \\
    \hline
    P2 $\rightarrow$ P1 & 2'b1x \\
    \hline
    \bottomrule
  \end{tabular}
  \caption{Multicast routing table of 3-port 32-bit router}
  \label{multicast_routing_table_3port_32bit_router}
\end{table}

%% \section{Router example: 32-bit 5-port}

%% \subsection{Packet structure}

%% \subsection{Parameters}

%% \subsection{Ports}

%% \subsection{Test environment}
